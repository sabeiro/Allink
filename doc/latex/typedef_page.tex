\-The potential energy functions used by \-M\-O\-L\-M\-C\-D cannot be chosen at run time, but can be easily modified at compile time, by redefining typedefs and recompiling. \-The potential energy classes used by \-Mol\-Mc\-D are referred to throughout the class library by a set of typedefs. \-Thus, for example, the class that represents a pair potential is referred to throughout the implementation of \-Md\-System and \-Mc\-System by a typedef \char`\"{}\-Pair\-Potential\char`\"{}. \-In the default configuration of \-Mol\-Mc\-D, \char`\"{}\-Pair\-Potential\char`\"{} is defined to be an alias for the \char`\"{}\-L\-J\-Potential\char`\"{} class.

\-In order to replace the \-Lennard-\/\-Jones potential by some other functional form for the pair potential, one must modify the typedef \char`\"{}\-Pair\-Potential\char`\"{} to refer to another pair potential class with the same interface, and then recompile the class library. \-The relevant typedefs for pair and bond potentials are defined in the header files src/base/\-Pair\-Potential.\-h and src/base/\-Bond\-Potential.\-h, respectively. \-Each of these typedef definition files also includes a header files for the actual pair and bond potential classes (e.\-g., \-L\-J\-Pair and \-Harmonic\-Bond), so that only class that includes the typedef also includes the header for the class to which it refers. \-See the documentation for these typedefs for a more detailed explanation of how to change potential classes by changing a few lines in the appropriate header.

\-We have chosen to use typedefs rather than polymorphic classes and virtual functions for the potential classes for reasons of efficiency. \-The force and energy evaluation methods are called repeatedly in the inner loops of \-M\-D and \-M\-C simulations, respectively. \-For efficiency, we would like to be able to inline these methods. \-The use of polymorphic classes with virtual force and energy evaluation methods would be more flexible, and would allow the user to choose potential energies at run time, without recompiling. \-It would also prevent inlining, however, and so would incur the full overhead associated with virtual function calls in the inner loop of either type of simulation.

\-The other candidate for similar treatment is the \-Boundary class. \-Boundary provides inline methods to evaluate separations and distances that are called within the inlined force and energy evaluation methods. \-At the moment, there is no need for a typedef because we have implemented only one version of the \-Boundary class, which represents a periodic orthorhombic unit cell. \-If and when we need the flexibility to define (for example) a more general triclinic unit cell, \-Boundary may be converted to a typedef in order to allow the user to select between orthorthombic and triclinic unit cells at compile time, while still allowing these functions to be inlined. 
\begin{DoxyItemize}
\item \hyperlink{extension_page}{\-Extending \-Mol\-Mc\-D by \-Inheritance} (\-Previous)  
\item \hyperlink{index}{\-Allink} (\-Up)  
\item \hyperlink{data_page}{\-Data \-Organization} (\-Next)  
\end{DoxyItemize}